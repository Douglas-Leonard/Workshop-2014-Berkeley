%%%%%%%%%%%%%%%%%%%%%%%%%%%%  PREAMBLE  %%%%%%%%%%%%%%%%%%%%%%%%%%%%%%%%
\documentclass{article}

%%  PACKAGES
\usepackage{amsthm}
\usepackage{enumerate}
\usepackage[leqno]{amsmath}
\usepackage{latexsym,amsfonts,amssymb}
\usepackage[all]{xy} \SelectTips{eu}{} \SilentMatrices
\usepackage[colorlinks=true,linkcolor=black,urlcolor=blue]{hyperref}
\usepackage[margin=1in]{geometry}

\newcommand{\bergroot}{\texttt{<bergmanroot>}}

\begin{document}

\begin{center}
\textsc{Installation instructions for NCAlgebras}
\end{center}

This system has been successfully installed on Linux (specifically Ubuntu)
as well as Mac OS X up through El Capitan.  We have yet to get it installed
on a Windows machine, except through a Linux VM.

\begin{enumerate}

\item Regardless of your platform, the first thing to do is install Macaulay2.
This can be done by following the instructions located at \url{www.macaulay2.com}.

\item Install Common Lisp.  On a Linux machine this is usually accomplished via
the package manager built into your distribution.  For example, on Ubuntu
the command \texttt{sudo apt-get install clisp} should do the trick.

On a Mac, one must first install homebrew.  Instructions for installing homebrew
are on their webpage \url{brew.sh}.  This installation may also require you to install
the XCode Command Line Developer Tools.  Instructions how to accomplish this are located
\href{http://osxdaily.com/2014/02/12/install-command-line-tools-mac-os-x/}{here}.
Once homebrew is installed, simply execute \texttt{brew install clisp} at a command prompt.

\item Download the Bergman system \href{http://servus.math.su.se/bergman/}{here}.
Extract the \texttt{tar.gz} file to a directory accessible by all the users that wish
to use the system.  In what follows, I will call the location of this directory \bergroot. 

\item Open a terminal, and navigate to the \bergroot~directory.  Here, the instructions
for Linux and Mac diverge a bit
\begin{itemize}
\item In Linux, change to the directory \bergroot/scripts/clisp/unix.  Execute the command
\texttt{./mkbergman -auto}.  This will build the bergman executable.  Move to step \ref{nextStep}.
\item On a Mac, things are a bit more complicated.  Change to the directory \bergroot/auxil/clisp.
Edit the file \texttt{bmtail-cl.lsp} in a text editor.  You will see the lines
\begin{verbatim}
;;(SAVEINITMEM "lispinit.mem" :INIT-FUNCTION...
(SAVEINITMEM "bergman.exe" :INIT-FUNCTION ...
\end{verbatim}
in the file.  In Common Lisp, \verb.;;. indicates a comment.  Switch the lines that are commented;
that is, place \verb.;;. on the front of the second line and take the \verb.;;. off the first line.
Save your changes.
\item Change to the \bergroot/scripts/clisp2.29/unix/ directory and execute the command:

\begin{center}\texttt{./mkbergman -auto}\end{center}

\item Finally, change directory to \bergroot/bin/clisp/unix.  In a text editor edit the \texttt{bergman} file there.
This is a shell script which loads the necessary files to start the bergman executable.  However, there
is a change that must be made to this file as well.  Here, \# denotes a comment.  By default
the third line is active and the second is commented.  Switch these around by uncommenting the second
line and commenting out the third line.  This ends the `special steps' required to install Bergman on a Mac.
\end{itemize}
\item \label{nextStep} Add a symbolic link from the path to the bergman executable script to /usr/local/bin.
This command should look something like

\begin{center}
\texttt{ln -s \bergroot/bin/clisp/unix/bergman /usr/local/bin/bergman},
\end{center}

provided you followed the instructions above to generated the bergman executable.  Note that in the above command,
the \emph{full} path (from the root directory) to \bergroot~must be given.
\item Add the line \texttt{"export BERGMANPATH=\bergroot"} to your init file for your shell.
\item Download NCAlgebra.m2 and NCAlgebraDoc.m2 from \href{http://users.wfu.edu/moorewf}{here} and place them in a fresh directory.  In this
directory, create a directory called NCAlgebra, and move the NCAlgebraDoc.m2 file into the NCAlgebra directory.
Alternatively, clone the git repository using the command

\begin{center}
\texttt{git clone https://github.com/Macaulay2/Workshop-2014-Berkeley.git}
\end{center}

which contains the guaranteed most recent version.  If you are unfamiliar with github then it is probably best
to download the files from the website.
\item Change to the directory containing NCAlgebra.m2, and start Macaulay2.  Run the command
\texttt{installPackage "NCAlgebra"} at the Macaulay2 prompt.
\item Test your installation by running the following commands in Macaulay2:
\begin{verbatim}
needsPackage "NCAlgebra"
R = fourDimSklyanin(QQ,{a,b,c,d})
hilbertBergman(R,DegreeLimit => 6)
\end{verbatim}
\item Enjoy!
\end{enumerate}

\end{document}
